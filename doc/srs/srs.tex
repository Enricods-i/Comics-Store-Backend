%Copyright 2014 Jean-Philippe Eisenbarth
%This program is free software: you can 
%redistribute it and/or modify it under the terms of the GNU General Public 
%License as published by the Free Software Foundation, either version 3 of the 
%License, or (at your option) any later version.
%This program is distributed in the hope that it will be useful,but WITHOUT ANY 
%WARRANTY; without even the implied warranty of MERCHANTABILITY or FITNESS FOR A 
%PARTICULAR PURPOSE. See the GNU General Public License for more details.
%You should have received a copy of the GNU General Public License along with 
%this program.  If not, see <http://www.gnu.org/licenses/>.

%Based on the code of Yiannis Lazarides
%http://tex.stackexchange.com/questions/42602/software-requirements-specification-with-latex
%http://tex.stackexchange.com/users/963/yiannis-lazarides
%Also based on the template of Karl E. Wiegers
%http://www.se.rit.edu/~emad/teaching/slides/srs_template_sep14.pdf
%http://karlwiegers.com
\documentclass{scrreprt}
\usepackage{listings}
\usepackage{underscore}
\usepackage[bookmarks=true]{hyperref}
\usepackage[utf8]{inputenc}
\usepackage[english]{babel}
\hypersetup{
    bookmarks=false,    % show bookmarks bar?
    pdftitle={Software Requirement Specification},    % title
    pdfauthor={Jean-Philippe Eisenbarth},              % author
    pdfsubject={TeX and LaTeX}, % subject of the document
    pdfkeywords={TeX, LaTeX, graphics, images}, % list of keywords
    colorlinks=true,       % false: boxed links; true: colored links
    linkcolor=blue,       % color of internal links
    citecolor=black,       % color of links to bibliography
    filecolor=black,        % color of file links
    urlcolor=purple,        % color of external links
    linktoc=page            % only page is linked
}
\def\myversion{1.0 }
\date{}
%\title
\usepackage{hyperref}
\begin{document}

\begin{flushright}
    \rule{16cm}{5pt}\vskip1cm
    \begin{bfseries}
        \Huge{DOCUMENTO DI SPECIFICA\\ DEI REQUISITI SOFTWARE}\\
        \vspace{1.9cm}
        Comics Store Back-End\\
        \vspace{1.9cm}
        Enrico De Salve\\
        \vspace{1.9cm}
    \end{bfseries}
\end{flushright}

\tableofcontents

\chapter{Introduzione}

\section{Propositi}
Il proposito di questo documento è quello di specificare i requisiti del sistema software
"Comics Store Back-End" per facilitarne la realizzazione e la validazione.

\section{Obiettivi}\label{obiettivi}
È necessaria un’applicazione web per l’implementazione di servizi relativi alla gestione 
di un negozio online associato ad una fumetteria.\\
\\
L'amministratore (il gestore della fumetteria) deve poter:
\begin{itemize}
    \item aggiungere (e rimuovere, se necessario) nuove collezioni di fumetti specificandone il 
    nome e l'immagine di copertina;
    \item creare (e rimuovere) nuove categorie di fumetti specificandone il nome ed una breve
    descrizione;
    \item associare le collezioni alle categorie, una collezione deve far parte di almeno una
    categoria, in generale può far parte di più categorie;
    \item aggiungere delle copie degli albi alle varie collezioni, si tenga presente che ogni
    fumetto deve far parte di una ed una sola collezione. Per aggiungere un nuovo fumetto nel 
    catalogo l'amministratore deve specificare il numero dell'albo, il prezzo unitario, lo 
    scrittore ed il disegnatore, il numero di pagine, il codice ISBN ed una descrizione;
    \item creare nuovi sconti specificando la percentuale di sconto e la data di scadenza;
    \item applicare uno sconto a più fumetti (non necessariamente appartenenti alla stessa 
    collezione);
    \item richiedere la generazione di report indicanti, in un certo intervallo di tempo, le
    vendite.
\end{itemize}
Gli utenti (anche se non registrati) possono ricercare le collezioni per titolo, scrittore,
disegnatore e codice ISBN; l'ordinamento delle collezioni ricercate deve poter essere deciso 
dall'utente. La ricerca per titolo deve mostrare anche le collezioni che corrispondono parzialmente 
all’input dell’utente.\\ 
Ad esempio se l’utente cerca la parola “attacco” l’applicazione deve mostrargli la collezione
chiamata “L’attacco dei giganti” ma anche “Star Wars: l’attacco dei cloni” e tutte le collezioni 
che contengano nel nome la parola “attacco”.\\
La ricerca mostra anche le collezioni contenenti fumetti che non sono disponibili per l'acquisto
(sono terminate le copie).\\
\\
Gli utenti possono registrarsi fornendo nome, cognome, data di nascita, indirizzo e-mail e
facoltativamente numero di telefono, paese e città in cui abitano.\\
Dal momento in cui si registra l’utente possiede un carrello.\\
\\
L'utente registrato che ha effettuato l'accesso deve poter:
\begin{itemize}
    \item aggiungere fumetti al suo carrello a patto che questi siano attualmente disponibili per 
    l’acquisto. Se esce dall’applicazione (effettua il log out) il contenuto del carrello deve
    essere memorizzato;
    \item rimuovere fumetti dal suo carrello;
    \item acquistare i fumetti contenuti nel suo carrello (se disponibili nel catalogo) attraverso
    la piattaforma Paypal, al termine dell’operazione il carrello deve essere svuotato;
    \item creare (ed eliminare, se già presenti) delle liste desideri. Ad ogni lista l’utente deve
    poter aggiungere e rimuovere qualunque fumetto (anche fumetti che, in quel momento, non sono 
    disponibili per l’acquisto);
    \item abilitare le notifiche per ogni lista desideri da lui creata: ogni cambiamento 
    riguardante prezzo, sconto e disponibilità di ogni fumetto contenuto nella lista gli deve essere
    notificato via email;
    \item modificare alcune delle informazioni inserite in fase di registrazione (indirizzo e-mail,
    numero di telefono, paese e città).
\end{itemize}

\section{Definizioni, acronimi ed abbreviazioni}
BE Back-End

\section{Riferimenti}
Nel documento "Progettazione base di dati" è riportata la struttura di memorizzazione dei dati.

\section{Panoramica}
La restante parte di questo documento contiene una descrizione dettagliata delle funzionalità richieste
al sistema software "Comics Store" secondo gli obiettivi espressi al punto \ref*{obiettivi}.


\chapter{Descrizione generale}

\section{Prospettive del prodotto}
Il sistema software "Comics Store BE" è la parte centrale di un'applicazione web (Comics Store per l'appunto),
rappresenta il componente che implementa la logica di business e deve fornire un interfaccia al componente client
dell'applicazione (il Front-End). L'interfaccia menzionata sarà una API RESTful.
Inoltre deve integrarsi con un DBMS che si occupa della memorizzazione e del reperimento dei dati.

\section{Funzionalità del prodotto}
Il sistema "Comics Store BE" deve:
\begin{itemize}
    \item Fornire la ricerca di collezioni di fumetti;
    \item Permettere solo agli utenti registrati di:
        \begin{itemize}
            \item aggiungere fumetti al carrello;
            \item rimuovere fumetti dal carrello;
            \item acquistare i fumetti presenti nel carrello;
            \item creare liste desideri;
            \item eliminare liste desideri;
            \item aggiungere fumetti ad una lista desideri;
            \item rimuovere fumetti da una lista desideri;
            \item modificare alcune delle suo informazioni personali;
        \end{itemize}
    \item Permettere solo all'amministratore di:
        \begin{itemize}
            \item aggiungere nuove collezioni;
            \item eliminare collezioni esistenti;
            \item creare nuove categorie;
            \item eliminare delle categorie;
            \item aggiungere delle categorie ad una collezione;
            \item rimuovere delle categorie da una collezione;
            \item aggiungere copie di fumetti al catalogo;
            \item aggiungere nuovi fumetti al catalogo;
            \item creare nuovi sconti;
            \item aggiungere uno sconto a dei fumetti;
            \item generare report delle vendite;
        \end{itemize}
    \item Salvare tutti gli acquisti effettuati dagli utenti
\end{itemize}

\section{Caratteristiche utente}
Il sistema software "Comics Store BE" è rivolto agli sviluppatori dell'interfaccia client che faranno
uso dei servizi forniti da questo sistema.

\section{Vincoli generali}
Il sistema deve rispettare i seguenti vincoli:
\begin{itemize}
    \item Un fumetto deve far parte di una ed una sola collezione
    \item Una collezione deve appartenere ad almeno una categoria
    \item Un utente non può aggiungere al carrello più copie di un fumetto di quante ce n'è siano disponibili
    \item Un utente non deve concludere l'acquisto se un fumetto non è più disponibile al momento dell'acquisto
\end{itemize}

\section{Assumptions and Dependencies}
Il sistema software "Comics Store" dovrà essere utilizzato


\chapter{External Interface Requirements}

\section{User Interfaces}
$<$Describe the logical characteristics of each interface between the software 
product and the users. This may include sample screen images, any GUI standards 
or product family style guides that are to be followed, screen layout 
constraints, standard buttons and functions (e.g., help) that will appear on 
every screen, keyboard shortcuts, error message display standards, and so on.  
Define the software components for which a user interface is needed. Details of 
the user interface design should be documented in a separate user interface 
specification.$>$

\section{Hardware Interfaces}
$<$Describe the logical and physical characteristics of each interface between 
the software product and the hardware components of the system. This may include 
the supported device types, the nature of the data and control interactions 
between the software and the hardware, and communication protocols to be 
used.$>$

\section{Software Interfaces}
$<$Describe the connections between this product and other specific software 
components (name and version), including databases, operating systems, tools, 
libraries, and integrated commercial components. Identify the data items or 
messages coming into the system and going out and describe the purpose of each.  
Describe the services needed and the nature of communications. Refer to 
documents that describe detailed application programming interface protocols.  
Identify data that will be shared across software components. If the data 
sharing mechanism must be implemented in a specific way (for example, use of a 
global data area in a multitasking operating system), specify this as an 
implementation constraint.$>$

\section{Communications Interfaces}
$<$Describe the requirements associated with any communications functions 
required by this product, including e-mail, web browser, network server 
communications protocols, electronic forms, and so on. Define any pertinent 
message formatting. Identify any communication standards that will be used, such 
as FTP or HTTP. Specify any communication security or encryption issues, data 
transfer rates, and synchronization mechanisms.$>$


\chapter{System Features}
$<$This template illustrates organizing the functional requirements for the 
product by system features, the major services provided by the product. You may 
prefer to organize this section by use case, mode of operation, user class, 
object class, functional hierarchy, or combinations of these, whatever makes the 
most logical sense for your product.$>$

\section{System Feature 1}
$<$Don’t really say “System Feature 1.” State the feature name in just a few 
words.$>$

\subsection{Description and Priority}
$<$Provide a short description of the feature and indicate whether it is of 
High, Medium, or Low priority. You could also include specific priority 
component ratings, such as benefit, penalty, cost, and risk (each rated on a 
relative scale from a low of 1 to a high of 9).$>$

\subsection{Stimulus/Response Sequences}
$<$List the sequences of user actions and system responses that stimulate the 
behavior defined for this feature. These will correspond to the dialog elements 
associated with use cases.$>$

\subsection{Functional Requirements}
$<$Itemize the detailed functional requirements associated with this feature.  
These are the software capabilities that must be present in order for the user 
to carry out the services provided by the feature, or to execute the use case.  
Include how the product should respond to anticipated error conditions or 
invalid inputs. Requirements should be concise, complete, unambiguous, 
verifiable, and necessary. Use “TBD” as a placeholder to indicate when necessary 
information is not yet available.$>$

$<$Each requirement should be uniquely identified with a sequence number or a 
meaningful tag of some kind.$>$

REQ-1:	REQ-2:

\section{System Feature 2 (and so on)}


\chapter{Other Nonfunctional Requirements}

\section{Performance Requirements}
$<$If there are performance requirements for the product under various 
circumstances, state them here and explain their rationale, to help the 
developers understand the intent and make suitable design choices. Specify the 
timing relationships for real time systems. Make such requirements as specific 
as possible. You may need to state performance requirements for individual 
functional requirements or features.$>$

\section{Safety Requirements}
$<$Specify those requirements that are concerned with possible loss, damage, or 
harm that could result from the use of the product. Define any safeguards or 
actions that must be taken, as well as actions that must be prevented. Refer to 
any external policies or regulations that state safety issues that affect the 
product’s design or use. Define any safety certifications that must be 
satisfied.$>$

\section{Security Requirements}
$<$Specify any requirements regarding security or privacy issues surrounding use 
of the product or protection of the data used or created by the product. Define 
any user identity authentication requirements. Refer to any external policies or 
regulations containing security issues that affect the product. Define any 
security or privacy certifications that must be satisfied.$>$

\section{Software Quality Attributes}
$<$Specify any additional quality characteristics for the product that will be 
important to either the customers or the developers. Some to consider are: 
adaptability, availability, correctness, flexibility, interoperability, 
maintainability, portability, reliability, reusability, robustness, testability, 
and usability. Write these to be specific, quantitative, and verifiable when 
possible. At the least, clarify the relative preferences for various attributes, 
such as ease of use over ease of learning.$>$

\section{Business Rules}
$<$List any operating principles about the product, such as which individuals or 
roles can perform which functions under specific circumstances. These are not 
functional requirements in themselves, but they may imply certain functional 
requirements to enforce the rules.$>$


\chapter{Other Requirements}
$<$Define any other requirements not covered elsewhere in the SRS. This might 
include database requirements, internationalization requirements, legal 
requirements, reuse objectives for the project, and so on. Add any new sections 
that are pertinent to the project.$>$

\section{Appendix A: Glossary}
%see https://en.wikibooks.org/wiki/LaTeX/Glossary
$<$Define all the terms necessary to properly interpret the SRS, including 
acronyms and abbreviations. You may wish to build a separate glossary that spans 
multiple projects or the entire organization, and just include terms specific to 
a single project in each SRS.$>$

\section{Appendix B: Analysis Models}
$<$Optionally, include any pertinent analysis models, such as data flow 
diagrams, class diagrams, state-transition diagrams, or entity-relationship 
diagrams.$>$

\section{Appendix C: To Be Determined List}
$<$Collect a numbered list of the TBD (to be determined) references that remain 
in the SRS so they can be tracked to closure.$>$

\end{document}

