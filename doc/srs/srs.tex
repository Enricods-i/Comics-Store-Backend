%Copyright 2014 Jean-Philippe Eisenbarth
%This program is free software: you can 
%redistribute it and/or modify it under the terms of the GNU General Public 
%License as published by the Free Software Foundation, either version 3 of the 
%License, or (at your option) any later version.
%This program is distributed in the hope that it will be useful,but WITHOUT ANY 
%WARRANTY; without even the implied warranty of MERCHANTABILITY or FITNESS FOR A 
%PARTICULAR PURPOSE. See the GNU General Public License for more details.
%You should have received a copy of the GNU General Public License along with 
%this program.  If not, see <http://www.gnu.org/licenses/>.

%Based on the code of Yiannis Lazarides
%http://tex.stackexchange.com/questions/42602/software-requirements-specification-with-latex
%http://tex.stackexchange.com/users/963/yiannis-lazarides
%Also based on the template of Karl E. Wiegers
%http://www.se.rit.edu/~emad/teaching/slides/srs_template_sep14.pdf
%http://karlwiegers.com
\documentclass{scrreprt}
\usepackage{listings}
\usepackage{underscore}
\usepackage[bookmarks=true]{hyperref}
\usepackage[utf8]{inputenc}
\usepackage[english]{babel}
\hypersetup{
    bookmarks=false,    % show bookmarks bar?
    pdftitle={Software Requirement Specification},    % title
    pdfauthor={Jean-Philippe Eisenbarth},              % author
    pdfsubject={TeX and LaTeX}, % subject of the document
    pdfkeywords={TeX, LaTeX, graphics, images}, % list of keywords
    colorlinks=true,       % false: boxed links; true: colored links
    linkcolor=blue,       % color of internal links
    citecolor=black,       % color of links to bibliography
    filecolor=black,        % color of file links
    urlcolor=purple,        % color of external links
    linktoc=page            % only page is linked
}
\def\myversion{1.0 }
\date{}
%\title
\usepackage{hyperref}
\begin{document}

\begin{flushright}
    \rule{16cm}{5pt}\vskip1cm
    \begin{bfseries}
        \Huge{DOCUMENTO DI SPECIFICA\\ DEI REQUISITI SOFTWARE}\\
        \vspace{1.9cm}
        Comics Store Back-End\\
        \vspace{1.9cm}
        Enrico De Salve\\
        \vspace{1.9cm}
    \end{bfseries}
\end{flushright}

\tableofcontents

\chapter{Introduzione}

\section{Propositi}
Il proposito di questo documento è quello di specificare i requisiti del sistema software
"Comics Store Back-End" per facilitarne la realizzazione e la validazione.

\section{Obiettivi}\label{obiettivi}
È necessaria un’applicazione web per l’implementazione di servizi relativi alla
gestione di un negozio online associato ad una fumetteria.\\
\\
Bisogna fornire al gestore della fumetteria la possibilità di aggiungere (e rimuovere) collezioni
di fumetti. Una collezione è caratterizzata da un titolo, da un'immagine di copertina e dai fumetti che 
contiene. Il gestore deve poter aggiungere nuovi fumetti specificandone il numero dell'albo, il prezzo
unitario, lo scrittore ed il disegnatore, il numero di pagine, il codice ISBN ed una descrizione.
Per ogni fumetto si memorizza la quantità di copie disponibili nel catalogo; deve, naturalmente, essere
possibile aggiungere copie di un fumetto incrementandone la quantità.\\
Ogni collezione fa parte di almeno una categoria, ma in generale ci possono essere più categorie associate
ad una collezione. Una collezione è caratterizzata da un nome e da una breve descrizione. Il gestore deve
poter creare ed eliminare le categorie ed associarle alle varie collezioni.\\
Il gestore della fumetteria deve poter, inoltre, creare nuovi sconti specificandone la percentuale di sconto
e la data di scadenza e deve poter applicarli a vari fumetti.\\
Infine, il gestore deve poter richiedere la generazione di report indicanti, per un certo intervallo di tempo,
le vendite.\\
\\
Gli utenti (anche se non registrati) possono ricercare le collezioni per titolo, scrittore,
disegnatore e codice ISBN; l'ordinamento delle collezioni ricercate deve poter essere deciso 
dall'utente. La ricerca per titolo deve mostrare anche le collezioni che corrispondono parzialmente 
all’input dell’utente.\\ 
Ad esempio se l’utente cerca la parola “attacco” l’applicazione deve mostrargli la collezione
chiamata “L’attacco dei giganti” ma anche “Star Wars: l’attacco dei cloni” e tutte le collezioni 
che contengano nel nome la parola “attacco”.\\
La ricerca mostra anche le collezioni contenenti fumetti che non sono disponibili per l'acquisto
(sono terminate le copie).\\
Gli utenti possono registrarsi fornendo nome, cognome, data di nascita, indirizzo e-mail e
facoltativamente numero di telefono, paese e città in cui abitano.\\
Dal momento in cui si registra l’utente possiede un carrello.\\
Un utente registrato che ha effettuato l'accesso ha a disposizione tutte le altre funzionalità offerte
dal sistema (oltre alla ricerca). Può aggiungere fumetti al suo carrello per poi acquistarli, naturalemente
i fumetti devono essere disponibili al momento dell'acquisto. Il contenuto del carrello deve essere memorizzato
cosi che quando l'utente effettua nuovamente l'accesso possa concludere l'acquisto, a seguito di un acquisto il
carrello viene svuotato.\\
Un utente registrato può creare (e a seguito di ciò eliminare) delle liste dei desideri, sono liste nelle quali
l'utente deve essere in grado di aggiungere qualunque fumetto (anche se non è attualmente disponibile nel catalogo).
Per ogni lista l'utente può scegliere se vuole essere notificato sulle modifiche relative ai fumetti presenti
nella lista desideri quali variazioni di prezzo, nuovi sconti o nuova disponobilità.\\
L'utente deve poter modificare alcune delle informazioni inserite in fase di registrazione (indirizzo e-mail,
numero di telefono, paese e città).

\section{Definizioni, acronimi ed abbreviazioni}
BE Back-End

\section{Riferimenti}
Nel documento "Progettazione base di dati" è riportata la struttura di memorizzazione dei dati.

\section{Panoramica}
La restante parte di questo documento contiene una descrizione dettagliata delle funzionalità richieste
al sistema software "Comics Store" secondo gli obiettivi espressi al punto \ref*{obiettivi}.


\chapter{Descrizione generale}

\section{Prospettive del prodotto}
Il sistema software "Comics Store BE" è la parte centrale di un'applicazione web (Comics Store per l'appunto),
rappresenta il componente che implementa la logica di business e che deve fornire un'interfaccia al componente client
dell'applicazione (il Front-End). L'interfaccia menzionata sarà una API REST.
Inoltre il sistema deve interagire con una base di dati che si occupa della memorizzazione e del reperimento dei dati.

\section{Funzionalità del prodotto}
Il sistema "Comics Store BE" deve:
\begin{itemize}
    \item Fornire la ricerca di collezioni di fumetti;
    \item Permettere solo agli utenti registrati di:
        \begin{itemize}
            \item aggiungere fumetti al carrello;
            \item rimuovere fumetti dal carrello;
            \item acquistare deii fumetti;
            \item creare liste desideri;
            \item eliminare liste desideri;
            \item aggiungere fumetti ad una lista desideri;
            \item rimuovere fumetti da una lista desideri;
            \item modificare alcune delle sue informazioni personali;
        \end{itemize}
    \item Permettere solo all'amministratore di:
        \begin{itemize}
            \item aggiungere nuove collezioni;
            \item eliminare collezioni esistenti;
            \item creare nuove categorie;
            \item eliminare delle categorie;
            \item aggiungere delle categorie ad una collezione;
            \item rimuovere delle categorie da una collezione;
            \item aggiungere copie di fumetti al catalogo;
            \item aggiungere nuovi fumetti al catalogo;
            \item creare nuovi sconti;
            \item aggiungere uno sconto a dei fumetti;
            \item generare report delle vendite;
        \end{itemize}
\end{itemize}

\section{Caratteristiche utente}
Il sistema software "Comics Store BE" è rivolto agli sviluppatori dell'interfaccia client che faranno
uso dei servizi forniti da questo sistema.

\section{Vincoli generali}
Riguardo all'inserimento nel catalogo di fumetti e collezioni bisogna garantire che un fumetto appartenga ad
\emph{una ed una sola} collezione e che una collezione appartenga ad \emph{almeno} una categoria.\\
Bisogna verificare che un utente non aggiunga più copie di un fumetto nel suo carrello di quante siano disponibili.\\
Prima di procedere all'acquisto dei prodotti contenuti nel carrello và verificato che per ognuno ci siano abbastanza
copie disponibili.


\chapter{Requisiti di interfaccia esterna}

\section{Interfaccia utente}
Il sistema software "Comics Store BE" deve essere dotato di un'interfaccia REST.

\section{Interfaccia hardware}
Il sistema software "Comics Store BE" non deve interfacciarsi con nessun sistema hardware.

\section{Interfaccia software}
Il sistema software "Comics Store BE" deve interfacciarsi con una base di dati per il salvataggio ed il reperimento
dei dati, è bene che si sia un livello di astrazione tra i due componenti in modo tale che si possa cambiare DBMS senza
alterare il funzionamento del sistema.\\
È anche necessario interfacciarsi con la piattaforma di pagamento Paypal che farà da intermediario durante l'acquisto.

\section{Interfaccia di comunicazione}
Il sistema software "Comics Store BE" deve comunicare con il lato client attraverso il protocollo HTTPS.\\
Il sistema deve comunicare con gli utenti anche via e-mail.


\chapter{Requisiti funzionali}

\section{Ricerca per titolo}\label{ric_titolo}

\subsection*{Introduzione}
Consente ad un utente di ricercare delle collezioni di fumetti per titolo.
\subsection*{Input}
Una sequenza di stringhe rappresentanti il titolo da ricercare.
\subsection*{Elaborazione}
Vengono visualizzate tutte le collezioni aventi un titolo che contiene in ordine tutte le stringhe della sequenza
in input. Per ciascuna collezione viene riportato il titolo e l'immagine di copertina.
\subsection*{Output}
L'elenco di collezioni risultato.

\section{Ricerca avanzata}

\subsection*{Introduzione}
Consente ad un utente di ricercare collezioni di fumetti per categoria, scrittore, disegnatore o codice ISBN.
\subsection*{Input}
I parametri della ricerca quindi il titolo, la categoria, lo scrittore, il disegnatore ed il codice ISBN.
In generale i parametri sono tutti opzionali ma almeno uno deve essere specificato.
\subsection*{Elaborazione}
Vengono visualizzate tutte le collezioni conformi ai parametri specificati in input. Per quanto riguarda
il titolo con conforme si intende come specificato nella sezione \ref*{ric_titolo}; tutti gli altri parametri
devono coincidere perfettamente con il campo corrispondente delle collezioni.
\subsection*{Output}
L'elenco di collezoini risultato.

\section{Aggiunta di un fumetto al carrello}

\subsection*{Introduzione}
Consente ad un utente registrato di aggiungere un fumetto al suo carrello.
\subsection*{Input}
Il fumetto da aggiungere e un identificativo dell'utente.
\subsection*{Elaborazione}
Se non ci sono copie del fumetto disponibili per l'acquisto nessuna aggiunta viene effettuata.
Se invece c'è almeno una copia disponibile e nel carrello il fumetto è già presente si incrementa
il numero di copie che l'utente intende acquistare, se il fumetto non è presente nel carrello si
aggiunge ad esso una copia.
\subsection*{Output}
La conferma dell'aggiunta.

\section{Rimozione di un fumetto dal carrello}

\subsection*{Introduzione}
Consente ad un utente registrato di rimuovere un fumetto dal suo carrello.
\subsection*{Input}
Il fumetto da rimuovere e un identificativo dell'utente.
\subsection*{Elaborazione}
Occorre verificare che ci siano copie del fumetto all'interno del carrello dell'utente.
Se c'è più di una copia viene decrementato il numero di copie del fumetto all'interno del carrello,
se c'è soltanto una copia il fumetto viene rimosso dal carrello. Se invece il fumetto non è presente
all'interno del carrello dell'utente non viene effettuata nessuna rimozione.
\subsection*{Output}
La conferma della rimozione.

\section{Acquisto dei fumetti}
DA FARE
\subsection*{Introduzione}
\subsection*{Input}
\subsection*{Elaborazione}
\subsection*{Output}

\section{Creazione lista desideri}

\subsection*{Introduzione}
Consente ad un utente registrato di creare una lista desideri.
\subsection*{Input}
Un identificativo dell'utente ed il nome della lista da creare.
\subsection*{Elaborazione}
Bisogna controllare che non esista già una lista creata con lo stesso nome creata dall'utente, in tal
caso la si crea.
\subsection*{Output}
La conferma dell'avvenuta creazione.

\section{Eliminazione lista desideri}

\subsection*{Introduzione}
Consente ad un utente registrato di eliminare una lista desideri precedentemente creata.
\subsection*{Input}
Un identificativo dell’utente e la lista da eliminare.
\subsection*{Elaborazione}
Occorre verificare che la lista esista ed in tal caso la si elimina insieme a tutto il suo contenuto.
\subsection*{Output}
La conferma dell'avvenuta eliminazione.

\section{}

\subsection*{Introduzione}
\subsection*{Input}
\subsection*{Elaborazione}
\subsection*{Output}


\chapter{Other Nonfunctional Requirements}

\section{Performance Requirements}
$<$If there are performance requirements for the product under various 
circumstances, state them here and explain their rationale, to help the 
developers understand the intent and make suitable design choices. Specify the 
timing relationships for real time systems. Make such requirements as specific 
as possible. You may need to state performance requirements for individual 
functional requirements or features.$>$

\section{Safety Requirements}
$<$Specify those requirements that are concerned with possible loss, damage, or 
harm that could result from the use of the product. Define any safeguards or 
actions that must be taken, as well as actions that must be prevented. Refer to 
any external policies or regulations that state safety issues that affect the 
product’s design or use. Define any safety certifications that must be 
satisfied.$>$

\section{Security Requirements}
$<$Specify any requirements regarding security or privacy issues surrounding use 
of the product or protection of the data used or created by the product. Define 
any user identity authentication requirements. Refer to any external policies or 
regulations containing security issues that affect the product. Define any 
security or privacy certifications that must be satisfied.$>$

\section{Software Quality Attributes}
$<$Specify any additional quality characteristics for the product that will be 
important to either the customers or the developers. Some to consider are: 
adaptability, availability, correctness, flexibility, interoperability, 
maintainability, portability, reliability, reusability, robustness, testability, 
and usability. Write these to be specific, quantitative, and verifiable when 
possible. At the least, clarify the relative preferences for various attributes, 
such as ease of use over ease of learning.$>$

\section{Business Rules}
$<$List any operating principles about the product, such as which individuals or 
roles can perform which functions under specific circumstances. These are not 
functional requirements in themselves, but they may imply certain functional 
requirements to enforce the rules.$>$


\chapter{Other Requirements}
$<$Define any other requirements not covered elsewhere in the SRS. This might 
include database requirements, internationalization requirements, legal 
requirements, reuse objectives for the project, and so on. Add any new sections 
that are pertinent to the project.$>$

\section{Appendix A: Glossary}
%see https://en.wikibooks.org/wiki/LaTeX/Glossary
$<$Define all the terms necessary to properly interpret the SRS, including 
acronyms and abbreviations. You may wish to build a separate glossary that spans 
multiple projects or the entire organization, and just include terms specific to 
a single project in each SRS.$>$

\section{Appendix B: Analysis Models}
$<$Optionally, include any pertinent analysis models, such as data flow 
diagrams, class diagrams, state-transition diagrams, or entity-relationship 
diagrams.$>$

\section{Appendix C: To Be Determined List}
$<$Collect a numbered list of the TBD (to be determined) references that remain 
in the SRS so they can be tracked to closure.$>$

\end{document}

